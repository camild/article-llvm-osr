\section{Related Work}
\label{se:related}

\paragraph{Early approaches}
OSR has been pioneered in the SELF language\cite{holzle1992self} to enable source-level debugging of optimized code, which required deoptimizing the code back to the original version. To reconstruct the source-level state, the compiler generates {\em scope descriptors} recording for each scope the locations or values of its arguments and locals. Execution can be interrupted only at certain interrupt points (i.e., method prologues and backward branches in loops) where its state is guaranteed to be consistent, allowing optimizations between interrupt points. It is worth mentioning also the {\em deferred compilation} mechanism\cite{chambers1991self} implemented in SELF for branches that are unlikely to occur at run-time: the system generates a stub which invokes the compiler to generate a code object that can reuse the stack frame of the original code.

\paragraph{Java Virtual Machines}
The success of the Java language has drawn more attention in the implementation of OSR techniques, as bytecode interpreters began to work along with JIT compilers. In the high-performance HotSpot Server JVM\cite{paleczny2001hotspot} performance-critical methods are identified using method-entry and backward-branches counters; when the OSR threshold is reached, the runtime transfers the execution from the interpreter frame to an OSR frame and the compiled code. Deoptimization is performed when class loading invalidates inlining or other optimization decisions: execution is rolled forward to a safe point, at which the native frame is converted into an interpreter frame.

Whaley\cite{whaley2001osr} proposes a technique to identify intra-method code regions that are rarely executed, and thus compile and optimize the code without these regions. A rare block is replaced by a stub that transfers the control to the interpreter, while a glue routine reconstructs the state from the interpreter starting from a table storing the location, in registers or memory, of each variable in the original bytecode.



\cite{fink2003design,detlefs2001method,soman2006efficient,lameed2013modular,steiner2007adaptive,chambers1992design}

  
  
  
  
  
  
  
  
  