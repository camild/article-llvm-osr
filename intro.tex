\section{Introduction}
\label{se:intro}

The LLVM compiler infrastructure provides a Just-In-Time compiler called MCJIT that is currently being used for generating optimized code at run-time in virtual machines for dynamic languages. MCJIT is employed in both industrial and research projects, including Webkit's Javascript engine, the open-source Python implementation Pyston, the Rubinius project for Ruby, Julia for high-performance technical computing, CXXR for the R language, Terra for Lua, and the Pure functional programming language. The MCJIT compiler shares the same optimization pipeline with static compilers such as clang, and it provides dynamic features such as native code loading and linking, as well as a customizable memory manager.

A piece that is currently missing in the environment is a feature to enable on-the-fly transitions between different versions of a running program's function. This feature is commonly known as On-Stack-Replacement (OSR) and is typically used in high-performance virtual machines, such as HotSpot and the Jikes RVM for Java, to interrupt a long-running function and recompile it at a higher optimization level. OSR can be a powerful tool for dynamic languages, for which most effective optimization decisions can typically be made only at run-time, when critical information such as type and shape of objects becomes available. In this scenario, OSR 

  