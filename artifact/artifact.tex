% LaTeX template for Artifact Evaluation V20151015
%
% Prepared by Grigori Fursin (cTuning foundation, France and dividiti, UK) 
% and Bruce Childers (University of Pittsburgh, USA)
%
% (C)opyright 2014-2015

\documentclass{sigplanconf}

\usepackage{url}

\newcommand{\osrkit}{{\sf OSRKit}}
\newcommand{\tinyvm}{{\sf TinyVM}}

\begin{document}

\special{papersize=8.5in,11in}

%%%%%%%%%%%%%%%%%%%%%%%%%%%%%%%%%%%%%%%%%%%%%%%%%%%%
% When adding this appendix to your paper, 
% please remove above part
%%%%%%%%%%%%%%%%%%%%%%%%%%%%%%%%%%%%%%%%%%%%%%%%%%%%

\appendix
\section{Artifact description}

%Submission and reviewing guidelines and methodology: \\
%{\em http://cTuning.org/ae/submission-20151015.html}

%%%%%%%%%%%%%%%%%%%%%%%%%%%%%%%%%%%%%%%%%%%%%%%%%%%%%%%%%%%%%%%%%%%%%
\subsection{Abstract}

\osrkit\ is a library that enables On-Stack Replacement (OSR) at arbitrary places in LLVM IR code. This artifact supports exploring how \osrkit\ can instrument IR code to support OSR transitions in the LLVM MCJIT runtime environment. A running example is presented based on the \texttt{isord} case study discussed in Section 3 of the companion paper. We also support repeating experiments Q1, Q2, and Q3 presented in Section 5. The artifact is based on a preconfigured Oracle VirtualBox VM.


%%%%%%%%%%%%%%%%%%%%%%%%%%%%%%%%%%%%%%%%%%%%%%%%%%%%%%%%%%%%%%%%%%%%%
\subsection{Description}

\subsubsection{Check-list (artifact meta information)}

%{\em Fill in whatever is applicable with some informal keywords and remove the rest}

{\small
\begin{itemize}
  %\item {\bf Algorithm: }
  \item {\bf Program: } {\tt shootout} C benchmarks (included, Sep 2015). %and a number of MATLAB benchmarks (included)
  \item {\bf Compilation: } LLVM 3.6.2 (release build).
  %\item {\bf Transformations: }
  %\item {\bf Binary: }
  %\item {\bf Data set: }
  \item {\bf Run-time environment: } Linux (version 3.x).
  \item {\bf Hardware: } x86-64 CPU.
  \item {\bf Run-time state: } Cache-sensitive (performance measurements only).
  %\item {\bf Execution: }
  \item {\bf Output: } Measures are output to console.
  \item {\bf Experiment workflow: } Invoke scripts and perform a few manual steps.
  \item {\bf Publicly available?} Yes.
\end{itemize}
}

\subsubsection{How delivered}

The artifact ships as an Oracle VirtualBox 5 Appliance.
The latest version of the code is available at \url{https://github.com/dcdelia/tinyvm}.

\subsubsection{Hardware dependencies}

The artifact requires an x86-64 platform.

\subsubsection{Software dependencies}

The artifact was tested in Oracle VirtualBox 5.0.10. 

%\subsubsection{Datasets}

%%%%%%%%%%%%%%%%%%%%%%%%%%%%%%%%%%%%%%%%%%%%%%%%%%%%%%%%%%%%%%%%%%%%%
\subsection{Installation}

To install the artifact, just import the appliance in Oracle VirtualBox.

%%%%%%%%%%%%%%%%%%%%%%%%%%%%%%%%%%%%%%%%%%%%%%%%%%%%%%%%%%%%%%%%%%%%%
\subsection{Experiment workflow}

We propose two usage sessions. In the first session, we show how to generate and instrument an LLVM IR code based on the \texttt{isord} example presented in Section 3 of the paper. The second session focuses on how to run the scripts used to generate the performance tables of Section 5 related to questions Q1, Q2, and Q3. Question Q4 is based on additional third-party software (the MATLAB McVM runtime) and is not addressed in this artifact.

%%%%%%%%%%%%%%%%%%%%%%%%%%%%%%%%%%%%%%%%%%%%%%%%%%%%%%%%%%%%%%%%%%%%%
\subsection{Evaluation and expected result}

\subsubsection{Session 1: OSR instrumentation in \osrkit}

\subsubsection{Session 2: Performance Figures}  

%%%%%%%%%%%%%%%%%%%%%%%%%%%%%%%%%%%%%%%%%%%%%%%%%%%%%%%%%%%%%%%%%%%%%
%\subsection{Notes}

%%%%%%%%%%%%%%%%%%%%%%%%%%%%%%%%%%%%%%%%%%%%%%%%%%%%
% When adding this appendix to your paper, 
% please remove below part
%%%%%%%%%%%%%%%%%%%%%%%%%%%%%%%%%%%%%%%%%%%%%%%%%%%%

\end{document}
