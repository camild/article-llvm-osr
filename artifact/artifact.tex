% !TEX root = ../article.tex

% LaTeX template for Artifact Evaluation V20151015
%
% Prepared by Grigori Fursin (cTuning foundation, France and dividiti, UK) 
% and Bruce Childers (University of Pittsburgh, USA)
%
% (C)opyright 2014-2015


%%%%%%%%%%%%%%%%%%%%%%%%%%%%%%%%%%%%%%%%%%%%%%%%%%%%
% When adding this appendix to your paper, 
% please remove above part
%%%%%%%%%%%%%%%%%%%%%%%%%%%%%%%%%%%%%%%%%%%%%%%%%%%%

\appendix
\section{Artifact Description}

%Submission and reviewing guidelines and methodology: \\
%{\em http://cTuning.org/ae/submission-20151015.html}

%%%%%%%%%%%%%%%%%%%%%%%%%%%%%%%%%%%%%%%%%%%%%%%%%%%%%%%%%%%%%%%%%%%%%
\subsection{Abstract}

\osrkit\ is a library that enables On-Stack Replacement (OSR) at arbitrary places in LLVM IR code. The artifact is designed to explore how \osrkit\ can instrument IR code to support OSR transitions in the LLVM MCJIT runtime environment. A running example is presented based on the \texttt{isord} case study discussed in Section 3. We also support repeating experiments Q1, Q2, and Q3 presented in Section 5. The artifact includes an interactive VM called \tinyvm\ for loading, inspecting, instrumenting, and executing IR code. The package is a preconfigured Oracle VirtualBox VM.


%%%%%%%%%%%%%%%%%%%%%%%%%%%%%%%%%%%%%%%%%%%%%%%%%%%%%%%%%%%%%%%%%%%%%
\subsection{Description}

The main component of the artifact is an interactive VM called \tinyvm\ built on top of the LLVM MCJIT runtime environment and the \osrkit\ library. The VM provides an interactive environment for IR manipulation, JIT-compilation, and execution of functions either generated at run-time or loaded from disk: for instance, it allows the user to insert OSR points in loaded functions, run optimization passes on them, display their CFGs, and repeatedly invoke a function for a specified amount of times. \tinyvm\ supports dynamic library loading and linking, and includes a helper component for MCJIT that simplifies tasks such as handling multiple IR modules, symbol resolution in presence of multiple versions of a function, and tracking native code and other machine-level generated objects such as Stackmaps.

\tinyvm\ is located in {\small\tt /home/osrkit/Desktop/tinyvm/} and runs a case-insensitite command-line interpreter:
\begin{small}
\begin{verbatim}
osrkit@osrkit-AE:~/Desktop/tinyvm$ tinyvm
Welcome! Enter 'HELP' to show the list of commands.
TinyVM> 
\end{verbatim}
\end{small}

\noindent Use ``help'' to print basic documentation on how to use the shell. Usage scenarios are discussed in \ref{ss:art-eval-res}.

\subsubsection{Check-list (artifact meta information)}

%{\em Fill in whatever is applicable with some informal keywords and remove the rest}

{\small
\begin{itemize}[parsep=0pt]
  %\item {\bf Algorithm: }
  \item {\bf Program: } {\tt shootout} C benchmarks (included, Sep 2015). %and a number of MATLAB benchmarks (included)
  \item {\bf Compilation: } LLVM 3.6.2 (release build).
  %\item {\bf Transformations: }
  %\item {\bf Binary: }
  %\item {\bf Data set: }
  \item {\bf Run-time environment: } Linux (version 3.x).
  \item {\bf Hardware: } x86-64 CPU.
  \item {\bf Run-time state: } Cache-sensitive (performance measurements only).
  %\item {\bf Execution: }
  \item {\bf Output: } Measures are output to console.
  \item {\bf Experiment workflow: } Invoke scripts and perform a few manual steps.
  \item {\bf Publicly available?} Yes.
\end{itemize}
}

\subsubsection{How Delivered}

The artifact ships as an Oracle VirtualBox 5 Appliance.
The latest version of the code is available at \url{https://github.com/dcdelia/tinyvm}.

\subsubsection{Hardware Dependencies}

An x86-64 platform is required.

\subsubsection{Software Dependencies}

The artifact was tested in Oracle VirtualBox 5.0.10. 

%\subsubsection{Datasets}

%%%%%%%%%%%%%%%%%%%%%%%%%%%%%%%%%%%%%%%%%%%%%%%%%%%%%%%%%%%%%%%%%%%%%
\subsection{Installation}

To install the artifact, just import the appliance in Oracle VirtualBox, which installs Linux LXLE. Open the {\tt README} file on the {\tt Desktop} folder for further info on the artifact and the Linux distribution.

%%%%%%%%%%%%%%%%%%%%%%%%%%%%%%%%%%%%%%%%%%%%%%%%%%%%%%%%%%%%%%%%%%%%%
\subsection{Experiment Workflow}

We propose two usage sessions. In the first session, we show how to generate and instrument an LLVM IR code based on the \texttt{isord} example presented in \mysection\ref{se:osr-llvm}. The second session focuses on how to run the scripts used to generate the performance tables of \mysection\ref{se:experiments} related to questions Q1, Q2, and Q3. Question Q4 is based on additional third-party software (the MATLAB McVM runtime) and is not addressed in the artifact.

%%%%%%%%%%%%%%%%%%%%%%%%%%%%%%%%%%%%%%%%%%%%%%%%%%%%%%%%%%%%%%%%%%%%%
\subsection{Evaluation and Expected Result}
\label{ss:art-eval-res}

% !TEX root = ../article.tex

\subsubsection{Session 1: OSR instrumentation in \osrkit}

\tinyvm\ implements a code generator for open OSR points that can dynamically inline function calls to targets that cannot be statically determined. In the example from \myfigure\ref{fi:isord-example}, a comparator function {\tt c} is passed as argument to function {\tt isord}, which checks whether an array {\tt v} of numbers is ordered according to the criterion encoded in {\tt c}.

To interactively reproduce the experiment presented in \mysection\ref{se:osr-llvm}, we provide under the folder {\small\tt tinyvm/isord} a C module {\small\tt inline.c} with an LLVM IR counterpart {\small\tt inline.ll} (generated with {\small\tt clang -S -emit-llvm -O1 inline.c}).

\vspace{0.2em}
We can load the IR module in \tinyvm\ and show the code generated for method {\tt isord} with:
\begin{small}
\begin{verbatim}
osrkit@osrkit-AE:~/Desktop/tinyvm$ tinyvm
Welcome! Enter 'HELP' to show the list of commands.
TinyVM> LOAD_IR isord/inline.ll
[LOAD] Opening "isord/inline.ll" as IR source file...
TinyVM> DUMP isord
[...]
\end{verbatim}
\end{small}

\noindent Displayed virtual register names and basic block labels will often differ from those reported in \myfigure\ref{fig:isordfrom}, which have been refactored for the sake of readability. In particular, the loop body of {\tt isord} will look like:

\begin{small}
\begin{verbatim}
.lr.ph:                          ; preds = %2, %0
  %i.01 = phi i64 [ %10, %2 ], [ 1, %0 ]
  %4 = getelementptr inbounds i64* %v, i64 %i.01
  %.sum = add nsw i64 %i.01, -1
  %5 = getelementptr inbounds i64* %v, i64 %.sum
  %6 = bitcast i64* %5 to i8*
  %7 = bitcast i64* %4 to i8*
  %8 = tail call i32 %c(i8* %6, i8* %7) #3
  [...]
\end{verbatim}
\end{small}

\noindent A $\phi$-node {\tt \%i.01} is used to represent the index of the {\tt for} loop from the C code, and is set to {\tt \%10} when reached from the loop header (basic block {\tt \%2}) {\em after} a loop iteration. In fact, as a result of {\small \tt -O1} optimizations, when {\tt n>1} execution jumps from the function entrypoint {\tt \%0} directly into the loop body, initializing the $\phi$-node with {\tt 1}. Comparator {\tt c} is invoked with a tail call, storing its return value into virtual register {\tt \%8}.

OSR points can be inserted with the {\tt INSERT\_OSR} command, which allows several combinations of features (see {\tt HELP} for an exhaustive list). In this session we will modify {\tt isord} so that when the loop body is entered for the first time, an OSR is aggressively fired:

\begin{small}
\begin{verbatim}
TinyVM> INSERT_OSR 100 ALWAYS OPEN UPDATE IN isord
                   AT %4 DYN_INLINE %c
\end{verbatim}
\end{small}

\noindent \tinyvm\ will {\tt UPDATE} the function in the following way: an {\tt ALWAYS}-true OSR condition is checked before executing instruction {\tt \%4} to fire an {\tt OPEN} OSR transition to the {\tt DYN\_INLINE} code generator, which will inline any indirect function call to the function pointer {\tt \%c}. We choose {\tt \%4} as location for the OSR as it is the first non-$\phi$ instruction in the loop body; we also hint the LLVM back-end through IR profiling metadata that firing an OSR is {\tt 100}\%-likely.

The IR will now look like:

\begin{small}
\begin{verbatim}
.lr.ph:                          ; preds = %2, %0
  %i.01 = phi i64 [ %10, %2 ], [ 1, %0 ]
  %alwaysOSR = fcmp true double 0.000000e+00,
                                0.000000e+00
  br i1 %alwaysOSR, label %OSR_fire,
                    label %OSR_split, !prof !1

OSR_split:                       ; preds = %.lr.ph
  %4 = getelementptr inbounds i64* %v, i64 %i.01
  %.sum = add nsw i64 %i.01, -1
  [...]

OSR_fire:                        ; preds = %.lr.ph
  %OSRCast = bitcast i32 (i8*, i8*)* %c to i8*
  %OSRRet = call i32 @isord_stub(i8* %OSRCast,
                i64* %v, i64 %n,
                i32 (i8*, i8*)* %c,
                i64 %i.01)
  ret i32 %OSRRet
\end{verbatim}
\end{small}

\noindent\osrkit\ has split the {\tt \%.lr.ph} block for the OSR condition, also adding an {\tt OSR\_fire} block to transfer the execution state to {\tt isord\_stub} and eventually return the {\tt OSRRet} value. 

We can now let {\tt isord} run on an array dynamically initialized from the {\tt driver} method, which takes as argument the array length to use. The method will also populate it with elements ordered for the comparator in use (see {\small\tt inline.c}). For instance, we can ask {\tt driver} to set up an array of $100000$ elements and run {\tt isord} on it:

\begin{small}
\begin{verbatim}
TinyVM> driver(100000)
Time spent in creating continuation function:
                         0.000252396 seconds
Address of invoked function: 140652750196768
Function being inlined: cmp
Elapsed CPU time: 0 m 0 s 3 ms 417 us 157 ns
               (that is: 0.003417157 seconds)
Evaluated to: 1
\end{verbatim}
\end{small}

\noindent The method returns $1$ as result, indicating that the vector is ordered. Compared to \myfigure\ref{fig:isordascto}, IR code generated for the OSR continuation function {\tt isordto} ({\tt DUMP isordto}) is slightly different as the MCJIT compiler detects that additional optimizations (e.g., loop strength reduction) are possible and performs them right away. We expect code generated for {\tt isord\_stub} to be identical up to renaming to the IR reported in \myfigure\ref{fig:isordstub}.

To show native code generated by the MCJIT back-end, we can run \tinyvm\ in a debugger with {\small\tt gdb tinyvm} and leverage the debugging interface of MCJIT. For instance, once {\tt driver} has been invoked, we can switch to the debugger with {\tt CTRL-Z} and display the x86-64 code for any compiled method with:
\begin{small}
\begin{verbatim}
(gdb) disas isordto
Dump of assembler code for function isordto:
   [Base address: 0x00007ffff7ff2000]
   <+0>:   mov    -0x8(%rdi,%rcx,8),%edx
   <+4>:   sub    (%rdi,%rcx,8),%edx
   <+7>:   xor    %eax,%eax
   <+9>:   test   %edx,%edx
   <+11>:  jg     0x7ffff7ff201a <isordto+26>
   <+13>:  inc    %rcx
   <+16>:  mov    $0x1,%eax
   <+21>:  cmp    %rsi,%rcx
   <+24>:  jl     0x7ffff7ff2000 <isordto>
   <+26>:  retq
End of assembler dump.
\end{verbatim}
\end{small}

\noindent To return to \tinyvm, we can use the {\tt fg} command of {\tt gdb}.
% !TEX root = ../article.tex

\subsubsection{Session 2: Performance Figures}

The experiments can be repeated by executing scripts on a selection of the \shootout\ benchmarks~\cite{shootout}. Each benchmark was compiled in {\tt clang} with both {\tt -O0} and {\tt -O1}. For each benchmark {\tt X}, {\tt tinyvm/shootout/X/} contains the unoptimized and optimized ({\tt -O1}) IR code, each in two versions:

\begin{itemize}[parsep=0pt]
\item {\tt bench} and {\tt bench-O1}: IR code of the benchmark;
%\item {\tt codeQuality}: IR code of the benchmark with the hottest loop instrumented with a never-firing OSR;
\item {\tt finalAlwaysFire} and {\tt finalAlwaysFire-O1}: IR code of the benchmark preprocessed by turning the body of the hottest loop into a separate function (see \ref{ss:experim-results}).
\end{itemize}

\noindent Each experiment runs a warm-up phase followed by 10 identical trials. We manually collected the figures from the console output and analyzed them, computing confidence intervals. We show how to run the code using {\tt n-body} as an example. Times reported in this section have been measured in VirtualBox on an Intel Core i7 platform, a different setup than the one discussed in \ref{ss:bench-setup}.

\paragraph{Question Q1.} The purpose of the experiment is assessing the impact on code quality due to the presence of OSR points.
The first step consists in generating figures for the baseline (uninstrumented) benchmark version:
\begin{small}
\begin{verbatim}
tinyvm$ tinyvm shootout/scripts/bench/n-body
\end{verbatim}
\end{small}

\noindent The script is as follows:

\begin{small}
\begin{verbatim}
LOAD_IR shootout/n-body/bench.ll
bench(50000000)
REPEAT 10 bench(50000000)
QUIT
\end{verbatim}
\end{small}

\noindent which loads the IR code, performs a warm-up execution of the benchmark, and then 10 repetitions. The experiment duration is $\approx1$m, with a time per trial $\approx5.725$s. 

The benchmark with the hottest loop instrumented with a never-firing OSR can be run as follows:

\begin{small}
\begin{verbatim}
tinyvm$ tinyvm shootout/scripts/codeQuality/n-body
\end{verbatim}
\end{small}

\noindent The script is as follows:

\begin{small}
\begin{verbatim}
LOAD_IR shootout/n-body/bench.ll
INSERT_OSR 5 NEVER OPEN UPDATE IN bench AT %8 CLONE
bench(50000000)
REPEAT 10 bench(50000000)
QUIT
\end{verbatim}
\end{small}

\noindent The experiment duration is $\approx1$m with a time per trial: $\approx5.673$s. The ratio $5.673/5.725=0.990$ for {\tt n-body} is slightly smaller than the one reported in \ref{fig:code-quality-base} on the Intel Xeon. The experiment for building \ref{fig:code-quality-O1} uses scripts in {\tt bench-O1} and {\tt codeQuality-O1}.

\paragraph{Question Q2.} This experiment assesses the run-time overhead of an OSR transition by measuring the duration of an always-firing OSR execution and of a never-firing OSR execution, and reporting the difference averaged over the number of fired OSRs. The script for this is:
\begin{small}
\begin{verbatim}
tinyvm$ tinyvm shootout/scripts/bench/n-body
\end{verbatim}
\end{small}

\paragraph{Question Q3.} 


%[Q2] What is the run-time overhead of an OSR transition, for instance to a clone of the running function?
%[Q3] What is the overhead of \osrkit\ for inserting OSR points and creating a stub or a continuation function?
%[Q4] What kind of benefits can we expect by using OSR in a production environment based on LLVM?



%%%%%%%%%%%%%%%%%%%%%%%%%%%%%%%%%%%%%%%%%%%%%%%%%%%%%%%%%%%%%%%%%%%%%
%\subsection{Notes}

%%%%%%%%%%%%%%%%%%%%%%%%%%%%%%%%%%%%%%%%%%%%%%%%%%%%
% When adding this appendix to your paper, 
% please remove below part
%%%%%%%%%%%%%%%%%%%%%%%%%%%%%%%%%%%%%%%%%%%%%%%%%%%%


