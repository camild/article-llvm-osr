% !TEX root = ../article.tex

\subsubsection{Session 1: OSR instrumentation in \osrkit}

\tinyvm\ implements a code generator for open OSR points that can dynamically inline function calls to targets that cannot be statically determined. In the example from \myfigure\ref{fi:isord-example}, a comparator function {\tt c} is passed as argument to function {\tt isord}, which checks whether an array {\tt v} of numbers is ordered according to the criterion encoded in {\tt c}.

To interactively reproduce the experiment presented in \mysection\ref{se:osr-llvm}, we provide under the folder {\small\tt tinyvm/isord} a C module {\small\tt inline.c} with an LLVM IR counterpart {\small\tt inline.ll} (generated with {\small\tt clang -S -emit-llvm -O1 inline.c}).

\vspace{0.2em}
We can load the IR module in \tinyvm\ and show the code generated for method {\tt isord} with:
\begin{small}
\begin{verbatim}
osrkit@osrkit-AE:~/Desktop/tinyvm$ tinyvm
Welcome! Enter 'HELP' to show the list of commands.
TinyVM> LOAD_IR isord/inline.ll
[LOAD] Opening "isord/inline.ll" as IR source file...
TinyVM> DUMP isord
[...]
\end{verbatim}
\end{small}

\noindent Displayed virtual register names and basic block labels will often differ from those reported in \myfigure\ref{fig:isordfrom}, which have been refactored for the sake of readability. In particular, the loop body of {\tt isord} will look like:

\begin{small}
\begin{verbatim}
.lr.ph:                          ; preds = %2, %0
  %i.01 = phi i64 [ %10, %2 ], [ 1, %0 ]
  %4 = getelementptr inbounds i64* %v, i64 %i.01
  %.sum = add nsw i64 %i.01, -1
  %5 = getelementptr inbounds i64* %v, i64 %.sum
  %6 = bitcast i64* %5 to i8*
  %7 = bitcast i64* %4 to i8*
  %8 = tail call i32 %c(i8* %6, i8* %7) #3
  [...]
\end{verbatim}
\end{small}

\noindent A $\phi$-node {\tt \%i.01} is used to represent the index of the {\tt for} loop from the C code, and is set to {\tt \%10} when reached from the loop header (basic block {\tt \%2}) {\em after} a loop iteration. In fact, as a result of {\small \tt -O1} optimizations, when {\tt n>1} execution jumps from the function entrypoint {\tt \%0} directly into the loop body, initializing the $\phi$-node with {\tt 1}. Comparator {\tt c} is invoked with a tail call, storing its return value into virtual register {\tt \%8}.

OSR points can be inserted with the {\tt INSERT\_OSR} command, which allows several combinations of features (see {\tt HELP} for an exhaustive list). In this session we will modify {\tt isord} so that when the loop body is entered for the first time, an OSR is aggressively fired:

\begin{small}
\begin{verbatim}
TinyVM> INSERT_OSR 100 ALWAYS OPEN UPDATE IN isord
                   AT %4 DYN_INLINE %c
\end{verbatim}
\end{small}

\noindent \tinyvm\ will {\tt UPDATE} the function in the following way: an {\tt ALWAYS}-true OSR condition is checked before executing instruction {\tt \%4} to fire an {\tt OPEN} OSR transition to the {\tt DYN\_INLINE} code generator, which will inline any indirect function call to the function pointer {\tt \%c}. We choose {\tt \%4} as location for the OSR as it is the first non-$\phi$ instruction in the loop body; we also hint the LLVM back-end through IR profiling metadata that firing an OSR is {\tt 100}\%-likely.

The IR will now look like:

\begin{small}
\begin{verbatim}
.lr.ph:                          ; preds = %2, %0
  %i.01 = phi i64 [ %10, %2 ], [ 1, %0 ]
  %alwaysOSR = fcmp true double 0.000000e+00,
                                0.000000e+00
  br i1 %alwaysOSR, label %OSR_fire,
                    label %OSR_split, !prof !1

OSR_split:                       ; preds = %.lr.ph
  %4 = getelementptr inbounds i64* %v, i64 %i.01
  %.sum = add nsw i64 %i.01, -1
  [...]

OSR_fire:                        ; preds = %.lr.ph
  %OSRCast = bitcast i32 (i8*, i8*)* %c to i8*
  %OSRRet = call i32 @isord_stub(i8* %OSRCast,
                i64* %v, i64 %n,
                i32 (i8*, i8*)* %c,
                i64 %i.01)
  ret i32 %OSRRet
\end{verbatim}
\end{small}

\noindent\osrkit\ has split the {\tt \%.lr.ph} block for the OSR condition, also adding an {\tt OSR\_fire} block to transfer the execution state to {\tt isord\_stub} and eventually return the {\tt OSRRet} value. 

We can now let {\tt isord} run on an array dynamically initialized from the {\tt driver} method, which takes as argument the array length to use. The method will also populate it with elements ordered for the comparator in use (see {\small\tt inline.c}). For instance, we can ask {\tt driver} to set up an array of $100000$ elements and run {\tt isord} on it:

\begin{small}
\begin{verbatim}
TinyVM> driver(100000)
Time spent in creating continuation function:
                         0.000252396 seconds
Address of invoked function: 140652750196768
Function being inlined: cmp
Elapsed CPU time: 0 m 0 s 3 ms 417 us 157 ns
               (that is: 0.003417157 seconds)
Evaluated to: 1
\end{verbatim}
\end{small}

\noindent The method returns $1$ as result, indicating that the vector is ordered. Compared to \myfigure\ref{fig:isordascto}, IR code generated for the OSR continuation function {\tt isordto} ({\tt DUMP isordto}) is slightly different as the MCJIT compiler detects that additional optimizations (e.g., loop strength reduction) are possible and performs them right away. We expect code generated for {\tt isord\_stub} to be identical up to renaming to the IR reported in \myfigure\ref{fig:isordstub}.

To show native code generated by the MCJIT back-end, we can run \tinyvm\ in a debugger with {\small\tt gdb tinyvm} and leverage the debugging interface of MCJIT. For instance, once {\tt driver} has been invoked, we can switch to the debugger with {\tt CTRL-Z} and display the x86-64 code for any compiled method with:
\begin{small}
\begin{verbatim}
(gdb) disas isordto
Dump of assembler code for function isordto:
   [Base address: 0x00007ffff7ff2000]
   <+0>:   mov    -0x8(%rdi,%rcx,8),%edx
   <+4>:   sub    (%rdi,%rcx,8),%edx
   <+7>:   xor    %eax,%eax
   <+9>:   test   %edx,%edx
   <+11>:  jg     0x7ffff7ff201a <isordto+26>
   <+13>:  inc    %rcx
   <+16>:  mov    $0x1,%eax
   <+21>:  cmp    %rsi,%rcx
   <+24>:  jl     0x7ffff7ff2000 <isordto>
   <+26>:  retq
End of assembler dump.
\end{verbatim}
\end{small}

\noindent To return to \tinyvm, we can use the {\tt fg} command of {\tt gdb}.